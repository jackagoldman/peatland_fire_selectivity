\section{Interpretation of Results: Fire Selectivity in Peatland Ecosystems}

The analysis of fire selectivity across various peatland landcover types reveals significant insights into how weather conditions (ISI: Initial Spread Index) and combustible fuel availability (BUI: Buildup Index) influence fire behaviour at different quantiles of selectivity. Using quantile regression models at $\tau = 0.5$ (median selectivity, representing typical fire behaviour) and $\tau = 0.9$ (high selectivity, representing extreme fire events), we examined the fixed effects and evaluated model performance via pinball loss.

\subsection{Key Findings from Fixed Effects}

\subsubsection{ISI Effects}
- At $\tau = 0.5$, ISI.mean shows significant positive associations with selectivity in several landcovers, including open\_bog ($p < 0.001$), total\_bog ($p < 0.001$), open\_permafrost ($p = 0.001$), and forested\_permafrost ($p = 0.011$). This suggests that drier and windier weather conditions enhance fire selectivity in these peatland types during typical fires.
- At $\tau = 0.9$, the effects are stronger and more widespread. Significant positive slopes are observed in open\_bog ($p < 0.001$), total\_bog ($p < 0.001$), total\_permafrost ($p < 0.001$), forested\_permafrost ($p < 0.001$), open\_permafrost ($p = 0.001$), and others. The coefficients are generally larger (e.g., open\_bog: 0.0099 vs. 0.0053), indicating that extreme fires are more sensitive to weather-driven spread.

\subsubsection{BUI Effects}
- At $\tau = 0.5$, BUI.mean effects are less pronounced, with significant negative associations in mineral ($p = 0.002$) and open\_rich\_fen ($p = 0.014$), suggesting that in typical fires, more combustible fuels do not increase selectivity in these areas or fires may still resist burning into these areas under these conditions.
- At $\tau = 0.9$, BUI.mean shows significant positive effects in most landcovers, including forested\_bog ($p < 0.001$), forested\_permafrost ($p < 0.001$), forested\_poor\_fen ($p < 0.001$), and many others. This implies that fuel dryness strongly amplifies selectivity during high-selectivity fires, potentially due to increased flammability of drier peat.

\subsection{Model Performance: Pinball Loss}
Pinball loss values are consistently lower at $\tau = 0.9$ compared to $\tau = 0.5$ across all landcovers (e.g., forested\_bog: 0.0066 vs. 0.0123), indicating better model fit for predicting extreme selectivity events. This heterogeneity suggests that the drivers of fire selectivity are not uniform; factors like ISI and BUI disproportionately influence rare, intense fires.

\subsection{Implications for Fire Selectivity and Ecology}
These results highlight the non-linear nature of fire selectivity in peatlands. At the median level ($\tau = 0.5$), fires exhibit moderate responsiveness to environmental conditions, allowing for some ecological resilience (or RESISTANCE). However, at $\tau = 0.9$, extreme fires show heightened sensitivity to dry weather and fuels, potentially leading to rapid landscape changes in vulnerable peatland types like bogs and permafrost areas.

Ecologically, this has profound implications:
- **Peatland Fire Regimes**: Peatlands, critical for carbon storage and biodiversity, may experience increased fire frequency and be selected for more under climate change, as drier conditions amplify selectivity for flammable landcovers.
- **Management Strategies**: Fire management should prioritize monitoring ISI and BUI in high-risk areas (e.g., open bogs), focusing on extreme fire prevention to mitigate carbon emissions and habitat loss.
- **Research Directions**: Future studies could explore interactions with other factors (e.g., vegetation structure) and validate these quantile-based insights with field data.

Overall, the analysis underscores that while typical fires maintain some balance, extreme events driven by synergistic weather and fuel conditions pose significant threats to peatland ecosystems.